\section{Introduction}
\label{sec:intro}

Today's software development is increasingly relying on software package libraries distributed through open source software (OSS) package managers (such as Cargo, npm, Maven and CRAN). Rather than writing software from scratch, developers often choose to depend on existing software packages.
At the same time, collaborative online development platforms like GitHub make software development an inherently social phenomenon~\cite{DabbishSTH12,Mens2019IEEESW}.

The collection of packages distributed by a software package manager forms a \emph{socio-technical} dependency network. Packages depend on other packages that are required for installing and deploying them. Software developers are \emph{technically} contributing to these packages (e.g., by making commits, pull requests to the package's git repository). Software developers are also \emph{socially} active (e.g., by commenting on the commit and pull request activities).
%Do you think its better to move the issue reporting activity to the social activities?
The phenomenon of socio-technical congruence (a.k.a. Conway's law)~\cite{Conway1968, Herbsleb1999} assumes
that the package dependency network structure and the communication structure of its community of contributors are tightly interwoven. Very little research focuses on such socio-technical congruence at the ecosystem level~\cite{Palyart2018TSE}, or studies show the congruence evolves over time~\cite{Cataldo2008}.

My PhD research project %started in June 2019, and
aims to empirically study, within evolving OSS packaging ecosystems, the socio-technical congruence between software package dependencies and the interaction patterns of package contributors. As such, I am to gain a better understanding of how this affects the health of the packaging ecosystem.
To do so %for package dependency networks of OSS package managers,
I will explore 3 research questions: 
%
% to find if there is some tendency or correlation between package dependency network and the  collaborator communication network:
%\noindent 
\textbf{RQ$_1$} \emph{How does the dependency network structure influence social activity?} %(E.g., are contributors more likely to be active in commenting on packages they depend on than on other packages?
%\noindent
 \textbf{RQ$_2$} \emph{How does the social activity of package contributors increase their likelihood to start/stop depending on this package?}
%\noindent
\textbf{RQ$_3$} \emph{How does the social activity of package contributors increase their likelihood to start/stop becoming technically active?}


In the first phase of my research, I will focus on packages distributed through the Cargo package manager and developed on GitHub, the biggest online collaborative development platform.
%
I will focus on different types of \emph{technical} development activity (e.g., commits, pull requests and issue reports through GitHub) and \emph{social} communication activity (e.g., commit comments, pull request comments and issue comments through GitHub). 
%
More research questions, activity types and packaging ecosystems will be explored in a later phase, on the basis of the results obtained for the Cargo case study. These follow-up questions will focus on the expected benefits that socio-technical congruence will have on the health of the packaging ecosystem and its community, such as increased productivity, responsiveness, contributor intake and retention.