\section{Methodology}

The \textsf{libraries.io} monitoring service provides access to package dependency metadata for 36 different package managers.
Among these, I have selected the Cargo package manager as a case study. Cargo is the default package manager for packages (a.k.a. ``crates") for the Rust programming language. Cargo is fairly recent (created in 2014), and the development history of all its packages is available on GitHub. 
% ALEX: That's not true, not all packages are available on GitHub. Replace "all its" by "most of its"? The good thing with Cargo being fairly recent is that most of its packages are developed on "recent platforms" (in contrast with, e.g., cpan where you can find packages on launchpad or sourceforge, or even packages whose cvs, bug tracker, discussions, etc. are spread in many different services.
Cargo is growing fast in number of packages, package releases, dependencies and contributors. 
% ALEX: You could even refer to our paper "An Empirical Comparison of Dependency Network Evolution in Seven Software Packaging Ecosystems"
By studying this package dependency network since its inception I hope to gain insights into how the ecosystem growth affects the socio-technical congruence over time.

I used libraries.io to extract the entire temporal evolution of the Cargo package dependency network. 
% ALEX: Strictly speaking, you didn't not "use" libraries.io. And you didn't extract the data ;) Instead, you relied on a data dump they provided after *they* extracted the data. 
For each package, I extracted metadata such as the package name, release number and release date, maintainer, package dependencies and their versioning constraints.
% ALEX: You also retrieved from these metadata the (optional) link to a repository. Then you filtered out the ones that were not available on github (either because there was no link, or a link pointing outside of GH, or a link pointing to GH but leading to a 404).
From the GitHub repository associated to each package I gathered relevant technical and social activity. 
% ALEX: "each package" is misleading, as not all packages have an associated GH repository
This information includes all contributors and their role, the ``social'' commenting activities they were involved in (separated into commit comments, pull request comments and issue comments), and the ``technical'' development activities they were involved in (commits, pull requests and issue requests).
\textbf{\color{red}@Mehdi: TO DO: Mention here how big is the extracted dataset: how many packages; how many package releases; how may package dependencies; how many distinct contributors; how many commits; how many comments and so on.}

To explore and analyse the extracted temporal socio-technical package dependency network, I use Python scripts and notebooks, and rely of existing data analytics and statistical libraries.

