\section{Methodology}

The \textsf{libraries.io} monitoring service provides access to package dependency metadata for 36 different package managers.
Among these, I have selected the Cargo package manager as a case study. Cargo is the default package manager for packages (a.k.a. ``crates") for the Rust programming language. Cargo is fairly recent (created in 2014), and the development history of all its packages is available on GitHub. Cargo is growing fast in number of packages, package releases, dependencies and contributors. By studying this package dependency network since its inception I hope to gain insights into how the ecosystem growth affects the socio-technical congruence over time.

I used libraries.io to extract the entire temporal evolution of the Cargo package dependency network. For each package, I extracted metadata such as the package name, release number and release date, maintainer, package dependencies and their versioning constraints.
From the GitHub repository associated to each package I gathered relevant technical and social activity. This information includes all contributors and their role, the ``social'' commenting activities they were involved in (separated into commit comments, pull request comments and issue comments), and the ``technical'' development activities they were involved in (commits, pull requests and issue requests).
\textbf{\color{red}@Mehdi: TO DO: Mention here how big is the extracted dataset: how many packages; how many package releases; how may package dependencies; how many distinct contributors; how many commits; how many comments and so on.}

To explore and analyse the extracted temporal socio-technical package dependency network, I use Python scripts and notebooks, and rely of existing data analytics and statistical libraries.

