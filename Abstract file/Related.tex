

\section{Background}

Several studies have been proposed about social aspects of software ecosystems for example how developer teams interact and evolve \cite{lopezfernandez2006:sna}, how newcomers progress in a software project \cite{Zhou2011-ICSE,Zhou2012-ICSE}, how the core team grows over time \cite{Robles2009MSR}, how developer teams get renewed \cite{Constantinou2017SANER}, and how socio-technical patterns affect software success or failure \cite{SurianTLCL13}.  Some study used Socio-technical metrics to define developer profiles and use them for specific recommendations \cite{Kintab:2014:RSE:2735522.2735526}, To reveal how bad social interaction practices can decrease software quality and team productivity \cite{Zanetti2013CHASE}. The notion of social debt has been used to evaluate the negative impact of community smells on industrial software \cite{Tamburri2015}. Team diversity factors have been shown to positively impact the team’s productivity \cite{Vasilescu2015CHI}. How social interaction occurs through issues and feature requests \cite{8049385}. 
