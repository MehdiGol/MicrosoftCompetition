

\section{Background}

Software ecosystems are large collections of interconnected software components with complex socio-technical interaction patterns \cite{Lungu2009,ManikasHansen2013}. They have become the norm in geographically distributed OSS  development. Typical well-studied ecosystems are software library package managers \cite{Decan2017SANER,Kikas2017,Dietrich2019} allowing to reuse software libraries for specific programming languages (e.g., npm, PyPI, CRAN, CPAN, RubyGems, Maven, Cargo). Their technical dependency networks grow at a rapid pace and may contain fragile packages that have a high transitive impact \cite{Decan2019EMSE}. 

Social issues are at least as important as technical ones~\cite{Mens2019IEEESW}. Social coding platforms can lead to effective work coordination strategies \cite{DabbishSTH12} and have become indispensable collaborative environments for software ecosystems \cite{Herbsleb1999}.
Researchers have studied the social aspects of how developer teams interact and evolve \cite{lopezfernandez2006:sna}, how newcomers progress in a software project \cite{Zhou2011-ICSE,Zhou2012-ICSE}, how the core team grows over time \cite{Robles2009MSR}, how developer teams get renewed \cite{Constantinou2017SANER}, and how socio-technical patterns affect software success or failure \cite{SurianTLCL13}. 

Social and technical issues are tightly interwoven and should be addressed conjointly, because of Conway's law stating that the software structure mimicks the communication and coordination structure of the community developing it \cite{Conway1968, Herbsleb1999, Kwan2011,Blincoe2019}. New models are required to better understand such socio-technical congruence at the ecosystem level \cite{Palyart2018TSE}. In addition, the temporal dimension needs to be taken into account since the contributor and technical relationships evolve over time \cite{Cataldo2008}.
Combining social and technical information leads to better prediction models, for example to detect faults in software components \cite{Bird2009,Bhattacharya2012}, to predict whether a project participant will become a developer~\cite{Gharehyazie2013-ICSM}, and to recommend software experts \cite{Kintab:2014:RSE:2735522.2735526}

