\section{ANALYSIS AND RESULTS}

To study the socio-technical congruence of Cargo, I started analysing its package dependency network and the associated technical and social (commenting) activities of its GitHub contributors.
%For this purpose I used Python notebooks and relied on existing data analytics and statistical libraries.
Focusing on the research questions of Section~\ref{sec:intro}, I report some very preliminary analysis below.
For now, the presented results are anecdotal and need to be complemented with proper statistical hypothesis testing, survival analysis and prediction modeling.

I started to investigate how the presence of commenting activity in a package repository relates to the introduction of a package dependency.
To do so, I considered all packages for which a new dependency was added at some point in time, and analysed whether commenting activity could be observed in the package repository \emph{before} or \emph{after} depending on these packages. Figure~\ref{fig:fig1} summarises the results. One can observe that in more cases commenting activity started after the creation of a package dependency than before adding the dependency. 
An important shift in behaviour can be observed around September 2017, where the number of packages with commenting activity before starting to depend on them is increasing and an inverse trend shift is observed for the packages with commenting activity  after starting to depend on them. Why this trend shift occurs remains an open question for now.

\begin{figure}[thb]
\vspace{-0.3cm}
    \includegraphics[width=0.9\columnwidth]{Photos/RQ21.pdf} 
    \caption{Number of package repositories with first commenting activity \emph{before} or \emph{after} starting to depend on a package.}
    \label{fig:fig1}
\end{figure}

In order to assess which types of comments are more likely to lead to the introduction of new package dependencies, I analysed the proportion of comment types for all comments made in repositories of packages prior to the addition of a dependency to those packages. Figure~\ref{fig:fig2} presents these results. 
One can observe that, among the four types of comments considered (i.e., commit comments, issue comments, pull request comments and pull request review comments), the proportion of comments on pull requests and issue requests is considerably higher than for commit comments and review comments. I hypothesise that commenting on pull requests and issue requests for a package could serve as a good predictor for adding new dependencies to that package.

\begin{figure}[htb]
    \includegraphics[width=0.9\columnwidth]{Photos/RQ22.pdf} 
    \caption{Proportion of comment types made in the repositories of packages before starting to depend on that package.}
    \label{fig:fig2}
\end{figure}

I also started to investigate whether social (i.e., commenting) activity on a package repository increases the likelihood to become technically active on that repository (i.e., submitting commits or pull requests). 
Figure~\ref{fig:fig3} presents some preliminary results. Considering the four different types of commenting activity, one can observe that issue comments are more likely to result in becoming technically active on a package repository.

\begin{figure}[thb]
    \includegraphics[width=0.9\columnwidth]{Photos/RQ3.pdf} 
    \caption{Number of comments (broken down by comment type) before starting  to technically contribute to a package.}
    \label{fig:fig3}
\end{figure}


